% when compiling this document and pushing, ensure that all the intermediate files are
% not pushed, e.g., set up the latex compiler to put everything in the output directory
% (which is ignored) or use overleaf and copy the source and pdf file

\documentclass[a4paper,english]{scrartcl}
\KOMAoptions{
  headlines=2.1,
  captions=tableheading,
  DIV=17,
  toc=bib,
  toc=listof,
  toc=sectionentrywithdots,
  listof=flat,  
  headings=optiontohead,
}
\addtokomafont{disposition}{\mdseries\slshape}
\addtokomafont{sectionentry}{\rmfamily\mdseries\normalfont}
\addtokomafont{section}{\Large}
\addtokomafont{subsection}{\Large}
\addtokomafont{subsubsection}{\large}
\addtokomafont{descriptionlabel}{\ttfamily}
\addtokomafont{footnote}{\slshape}
\addtokomafont{footnotelabel}{\bfseries}
\addtokomafont{footnotereference}{\bfseries}
\addtokomafont{itemizelabel}{\bfseries}
\addtokomafont{labelitemi}{\bfseries}
\addtokomafont{labelinglabel}{\bfseries}
\addtokomafont{labelingseparator}{\rmfamily}
\addtokomafont{pageheadfoot}{\slshape}
\addtokomafont{caption}{\slshape}
\addtokomafont{captionlabel}{\upshape\bfseries}
\RedeclareSectionCommand[tocnumwidth=1.30em]{section}
\RedeclareSectionCommand[tocnumwidth=2.05em, tocindent=1.3em]{subsection}
\RedeclareSectionCommand[tocnumwidth=2.80em, tocindent=3.35em]{subsubsection}
\DeclareTOCStyleEntry[beforeskip=3pt plus .2pt]{section}{section}
\setcounter{secnumdepth}{3}
\setcounter{tocdepth}{3}
\RequirePackage{setspace}
\setstretch{1.065}
\setlength{\parindent}{0.4cm}
\setlength{\parskip}{0.07cm plus 0.013cm minus 0.06cm}
\RequirePackage[autooneside=false]{scrlayer-scrpage}
\pagestyle{scrheadings}
\clearpairofpagestyles
\addtokomafont{pageheadfoot}{\small}
\renewcommand*{\sectionmarkformat}{}
\automark[subsection]{section}
\automark*[subsubsection]{}
\ihead{\leftmark}
\ohead{\ifthenelse{\equal{\leftmark}{\rightmark}}{}{\rightmark}}
\cfoot*{\pagemark}
\KOMAoptions{headsepline=0.1pt}
\AfterEndEnvironment{figure}{\noindent\ignorespaces}
\AfterEndEnvironment{table}{\noindent\ignorespaces}

\RequirePackage[T1]{fontenc}
\RequirePackage{lmodern}
\RequirePackage{babel}
\RequirePackage{microtype}
\binoppenalty=700
\brokenpenalty=10000
\clubpenalty=10000
\displaywidowpenalty=10000
\exhyphenpenalty=1000
\floatingpenalty=20000
\hyphenpenalty=60
\interlinepenalty=0
\linepenalty=10
\postdisplaypenalty=0
\predisplaypenalty=10000
\relpenalty=500
\widowpenalty=10000
\interfootnotelinepenalty=10000
\frenchspacing    
\RequirePackage[shortcuts]{extdash}

\RequirePackage[
  style=nature_custom,
  citestyle=numeric-comp,
  isbn=false,
  sorting=none,
]{biblatex}
\RequirePackage{csquotes}   % for language specific quotation in literature
% (must be loaded after, e.g., scrreprt)

\RequirePackage{xcolor}
\definecolor{urlcol}{RGB}{0, 102, 204}
\RequirePackage{hyperref}
\hypersetup{colorlinks=true,urlcolor=urlcol,linkcolor=black,citecolor=blue,breaklinks}
\RequirePackage{enumitem}
\setenumerate{listparindent=\parindent,label=(\alph*)}
\RequirePackage{float}
\RequirePackage{booktabs}
\heavyrulewidth=1.1pt

\RequirePackage{mathtools} % has to be loaded before cleveref
\usepackage{amsthm} % load before cleveref for proper referencing
\RequirePackage[english,capitalize]{cleveref}
\newtheoremstyle{standard}{1.0ex}{1.5ex}{\slshape}{}{\bfseries}{}{.7em}{}
\newtheoremstyle{statement}{1.0ex}{1.5ex}{\itshape}{}{\bfseries}{}{.7em}{}
% this has to be defined after cleveref for proper referencing
\theoremstyle{standard}
\newtheorem{definition}{Definition}[subsection]
\newtheorem{remark}[definition]{Remark}
\theoremstyle{statement}
\newtheorem{definition_proposition}[definition]{Definition and Proposition}
\newtheorem{lemma}[definition]{Lemma}
\newtheorem{proposition}[definition]{Proposition}
\newtheorem{proposition_definition}[definition]{Proposition and Definition}
\newtheorem{corollary}[definition]{Corollary}
\newtheorem{theorem}[definition]{Theorem}
\makeatletter
\renewenvironment{proof}[1][\proofname]{%
  \par%
  \vspace{-1.9ex}%
  \pushQED{\qed}%
  \normalfont\topsep1\p@\@plus1\p@\relax%
  \trivlist%
  \item[\hskip\labelsep\itshape#1\@addpunct{.}]%
  \ignorespaces%
}{\popQED\endtrivlist\@endpefalse}
\newenvironment{interrupted_proof}[1][\proofname]{%
  \par%
  \pushQED{\qed}%
  \normalfont \topsep1\p@\@plus1\p@\relax%
  \trivlist%
  \item[\hskip\labelsep \itshape #1\@addpunct{.}]%
  \ignorespaces%
}{\popQED\endtrivlist\@endpefalse}
\AfterEndEnvironment{definition}{\@doendpe}
\AfterEndEnvironment{lemma}{\@doendpe}
\AfterEndEnvironment{proposition}{\@doendpe}
\AfterEndEnvironment{proposition_definition}{\@doendpe}
\AfterEndEnvironment{corollary}{\@doendpe}
\AfterEndEnvironment{theorem}{\@doendpe}
\AfterEndEnvironment{proof}{\@doendpe}
\AfterEndEnvironment{interrupted_proof}{\@doendpe}
\makeatother
\crefname{section}{Sec.}{Secs.}%
\crefname{appendix}{App.}{Apps.}%
\crefname{definition}{Def.}{Defs.}%
\crefname{remark}{Rem.}{Rems.}%
\crefname{definition_proposition}{Def.}{Defs.}%
\crefname{lemma}{Lem.}{Lems.}%
\crefname{proposition}{Prop.}{Props.}%
\crefname{proposition_definition}{Prop.}{Props.}%
\crefname{corollary}{Cor.}{Cors.}%
\crefname{theorem}{Thm.}{Thms}%
\Crefname{section}{Section}{Sections}%
\Crefname{appendix}{Appendix}{Appendices}%
\Crefname{definition_proposition}{Definition}{Definitions}%
\Crefname{proposition}{Proposition}{Propositions}%
\Crefname{theorem}{Theorem}{Theorems}%

\RequirePackage{amssymb}
\RequirePackage{siunitx}
\RequirePackage{tensor}
\RequirePackage{icomma}
\RequirePackage{bm}
\usepackage{IEEEtrantools}
\RequirePackage{dsfont}
\RequirePackage{isomath}
\RequirePackage[scr=rsfs]{mathalpha}
\DeclareFontFamily{U}{BOONDOX-cal}{\skewchar\font=45}
\DeclareFontShape{U}{BOONDOX-cal}{m}{n}{
  <-> s*[1.08] BOONDOX-r-cal}{}
\DeclareFontShape{U}{BOONDOX-cal}{b}{n}{
  <-> s*[1.08] BOONDOX-b-cal}{}
\DeclareMathAlphabet{\mathboondox}{U}{BOONDOX-cal}{m}{n}
\SetMathAlphabet{\mathboondox}{bold}{U}{BOONDOX-cal}{b}{n}
\DeclareFontFamily{U}{dutchcal}{\skewchar\font=45}
\DeclareFontShape{U}{dutchcal}{m}{n}{
  <-> \mathalfa@calscaled  dutchcal-r}{}
\DeclareFontShape{U}{dutchcal}{b}{n}{
  <-> \mathalfa@calscaled  dutchcal-b}{}
\DeclareMathAlphabet{\mathdutchcal}{U}{dutchcal}{m}{n}
\SetMathAlphabet{\mathdutchcal}{bold}{U}{dutchcal}{b}{n}

\RequirePackage{xparse}
\RequirePackage{xspace}

% use the following commands, or their shortcuts below, instead of manually settting
% parentheses, braces, ...; they have better typesetting
\DeclarePairedDelimiter\Parenthesis{\lparen}{\rparen}
\DeclarePairedDelimiter\Brackets{\lbrack}{\rbrack}
\DeclarePairedDelimiter\Braces{\lbrace}{\rbrace}
\DeclarePairedDelimiter\AbsoluteValue{\lvert}{\rvert}
\DeclarePairedDelimiter\Norm{\lVert}{\rVert}
\DeclarePairedDelimiter\Bra{\langle}{\rvert}
\DeclarePairedDelimiter\Ket{\lvert}{\rangle}
\DeclarePairedDelimiter\BraKet{\langle}{\rangle}
\DeclarePairedDelimiterX\BraKetTwo[2]{\langle}{\rangle}{%
  #1\delimsize\vert\mathopen{}#2%
}
\DeclarePairedDelimiterX\BraKetThree[3]{\langle}{\rangle}{%
  #1\delimsize\vert\mathopen{}#2%
  \delimsize\vert\mathopen{}#3%
}
\DeclarePairedDelimiterX\BraKetTwoStar[2]{\langle}{\rangle}{%
  #1\nonscript\,\delimsize\vert\nonscript\,\mathopen{}#2}
\DeclarePairedDelimiterX\BraKetThreeStar[3]{\langle}{\rangle}{%
  #1\nonscript\,\delimsize\vert\nonscript\,\mathopen{}#2\nonscript\,%
  \delimsize\vert\nonscript\,\mathopen{}#3%
}
\DeclarePairedDelimiter\DotDot{.}{.}
\DeclarePairedDelimiter\ParenthesisDot{\lparen}{.}
\DeclarePairedDelimiter\DotParenthesis{.}{\rparen}
\DeclarePairedDelimiter\BraceDot{\lbrace}{.}
\DeclarePairedDelimiter\DotBrace{.}{\rbrace}
\DeclarePairedDelimiter\ParenthesisBracket{\lparen}{\rbrack}
\DeclarePairedDelimiter\BracketParenthesis{\lbrack}{\rparen}
\DeclarePairedDelimiterX\SetVerticalLine[2]{\lbrace}{\rbrace}{%
  #1\nonscript\:\delimsize\vert\allowbreak\nonscript\:\mathopen{}#2%
}
\DeclarePairedDelimiterX\SetColon[2]{\lbrace}{\rbrace}{%
  #1\nonscript\::\allowbreak\nonscript\:\mathopen{}#2%
}
\DeclarePairedDelimiterX\GroupSetVerticalLine[2]{\langle}{\rangle}{%
  #1\nonscript\:\delimsize\vert\allowbreak\nonscript\:\mathopen{}#2%
}
\DeclarePairedDelimiterX\GroupSetColon[2]{\langle}{\rangle}{%
  #1\nonscript\::\allowbreak\nonscript\:\mathopen{}#2%
}
% shortcuts
\NewDocumentCommand\pa{sm}{\IfBooleanTF{#1}{\Parenthesis{#2}}{\Parenthesis*{#2}}}
\NewDocumentCommand\bk{sm}{\IfBooleanTF{#1}{\Brackets{#2}}{\Brackets*{#2}}}
\NewDocumentCommand\bc{sm}{\IfBooleanTF{#1}{\Braces{#2}}{\Braces*{#2}}}
\NewDocumentCommand\abs{sm}{\IfBooleanTF{#1}{\AbsoluteValue{#2}}{\AbsoluteValue*{#2}}}
\NewDocumentCommand\norm{sm}{\IfBooleanTF{#1}{\Norm{#2}}{\Norm*{#2}}}
\NewDocumentCommand\dotdot{sm}{\IfBooleanTF{#1}{\DotDot{#2}}{\DotDot*{#2}}}
\NewDocumentCommand\padot{sm}{\IfBooleanTF{#1}{\ParenthesisDot{#2}}{\ParenthesisDot*{#2}}}
\NewDocumentCommand\dotpa{sm}{\IfBooleanTF{#1}{\DotParenthesis{#2}}{\DotParenthesis*{#2}}}
\NewDocumentCommand\bcdot{sm}{\IfBooleanTF{#1}{\BraceDot{#2}}{\BraceDot*{#2}}}
\NewDocumentCommand\dotbc{sm}{\IfBooleanTF{#1}{\DotBrace{#2}}{\DotBrace*{#2}}}
\NewDocumentCommand\bkpa{sm}{\IfBooleanTF{#1}{\BracketParenthesis{#2}}%
  {\BracketParenthesis*{#2}}}
\NewDocumentCommand\pabk{sm}{\IfBooleanTF{#1}{\ParenthesisBracket{#2}}%
  {\ParenthesisBracket*{#2}}}
\NewDocumentCommand\bra{smd""}{%
  \IfBooleanTF{#1}{\Bra{#2}\IfValueTF{#3}{_{#3}}{}}{\Bra*{#2}\IfValueTF{#3}{_{#3}}{}}%
}
\NewDocumentCommand\ket{smd""}{%
  \IfBooleanTF{#1}{\Ket{#2}\IfValueTF{#3}{_{#3}}{}}{\Ket*{#2}\IfValueTF{#3}{_{#3}}{}}%
}
\NewDocumentCommand\braket{smd""ood""d>>}{%
  {}\IfValueTF{#7}{_{\IfValueTF{#3}{#3}{#7}}}{\IfValueTF{#3}{_{#3}}{}}%
  \IfBooleanTF{#1}%
  {%
    \IfValueTF{#4}{\IfValueTF{#5}{\BraKetThree{#2}{#4}{#5}}{\BraKetTwo{#2}{#4}}}%
    {\BraKet{#2}}%
  }%
  {%
    \IfValueTF{#4}{\IfValueTF{#5}{\BraKetThreeStar*{#2}{#4}{#5}}{\BraKetTwoStar*{#2}{#4}}}%
    {\BraKet*{#2}}%
  }%
  \IfValueTF{#7}{_{\IfValueTF{#6}{#6}{#7}}}{\IfValueTF{#6}{_{#6}}{}}%
}
\NewDocumentCommand\ketbra{smd""od""d>>}{%
  \IfBooleanTF{#1}%
  {%
    \Ket{#2}\IfValueTF{#6}{_{\IfValueTF{#3}{#3}{#6}}}%
    {\IfValueTF{#3}{_{#3}}{}}\!\Bra*{\IfValueTF{#4}{#4}{#2}}%
  }%
  {%
    \Ket*{#2}\IfValueTF{#6}{_{\IfValueTF{#3}{#3}{#6}}}%
    {\IfValueTF{#3}{_{#3}}{}}\!\Bra*{\IfValueTF{#4}{#4}{#2}}%
  }%
  \IfValueTF{#6}{_{\IfValueTF{#5}{#5}{#6}}}{\IfValueTF{#5}{_{#5}}{}}%
}
\let\Set\SetVerticalLine % \let\Set\SetColon as alternative
\let\GroupSet\GroupSetVerticalLine
\newcommand\set[2]{\Set*{#1}{#2}}
\newcommand\groupset[2]{\GroupSet*{#1}{#2}}
% subequation and align in one
\newenvironment{subalign}{\subequations\align}{\endalign\endsubequations}
% matrices
\newcommand\mat[1]{\begin{matrix}#1\end{matrix}}
\newcommand\matp[1]{\begin{pmatrix}#1\end{pmatrix}}
\newcommand\matb[1]{\begin{bmatrix}#1\end{vmatrix}}
\newcommand\matB[1]{\begin{Bmatrix}#1\end{Bmatrix}}
\newcommand\matv[1]{\begin{vmatrix}#1\end{vmatrix}}
\newcommand\matV[1]{\begin{Vmatrix}#1\end{Vmatrix}}
\newcommand\smat[1]{\begin{smallmatrix}#1\end{smallmatrix}}
\newcommand\smatp[1]{\Parenthesis*{\begin{smallmatrix}#1\end{smallmatrix}}}

\newcommand\minusset{\!\setminus\!}

% now some specific definitions and shortcuts we are using
\DeclareMathOperator\tr{Tr}
\DeclareMathOperator\id{id}
% Hilbert space, basis, ..
\newcommand*\hi{\mathcal{H}} % hilbert
\newcommand*\ba{\mathcal{B}} % basis
\newcommand*\ls{\mathscr{L}} % linear stetig
% \newcommand*\LL{\mathcal{L}}
\newcommand*\bh{\mathboondox{Q}} % for qubits space / (b)inary (h)ilbert
\newcommand*\de{\mathboondox{D}} % density operator/matrix

% Some common sets, groups, ...
\newcommand*\N{\mathbb{N}}
\newcommand*\No{\mathbb{N}_0}
\newcommand*\Z{\mathbb{Z}}
\newcommand*\Q{\mathbb{Q}}
\newcommand*\R{\mathbb{R}}
\newcommand*\C{\mathbb{C}}
\newcommand*\F{\mathbb{F}}
\newcommand*\K{\mathbb{K}}
\newcommand*\M{\mathrm{M}}
\newcommand*\U{\mathrm{U}}
\newcommand*\SU{\mathrm{SU}}
\newcommand*\GL{\mathrm{GL}}
\newcommand*\Sp{\mathrm{Sp}}

% correct space when writing equations over multiple lines
\newcommand*\eqspace{\mathrel{\phantom{=}}}

% common notations, that may take an argument or not
\NewDocumentCommand\order{o}{\mathcal{O}\IfValueTF{#1}{\pa{#1}}{}}
\let\olddim\dim
\RenewDocumentCommand\dim{o}{\olddim\IfValueTF{#1}{\pa{#1}}{}}
\NewDocumentCommand\aut{m}{\mathrm{Aut}\pa{#1}}
\NewDocumentCommand\cen{o}{\mathrm{Z}\IfValueTF{#1}{\pa{#1}}{}}
\NewDocumentCommand\ces{od""}{\mathrm{C}\IfValueTF{#2}{_{#2}}{}\IfValueTF{#1}{\pa{#1}}{}}
\NewDocumentCommand\nos{od""}{\mathrm{N}\IfValueTF{#2}{_{#2}}{}\IfValueTF{#1}{\pa{#1}}{}}
% \NewDocumentCommand\nos{om}{\mathrm{N}\IfValueTF{#1}{_{#1}}{}\pa{#2}}

% other
\newcommand*\ee{\mathrm{e}}
\let\da\dagger
\newcommand*\ur{^\dagger}
\newcommand\ts{\intercal} % fix that for consistency \newcommand\ts{T}
\newcommand*\ut{^\ts}
\newcommand*\ui{^{-1}}
\newcommand*\inn{\mathrm{inn}}
\newcommand*\1{\mathds{1}}
\newcommand*\E{\mathcal{E}} % "big varepsilon"
\newcommand*\cc{\mathrm{c.c.}}
\newcommand*\hc{\mathrm{h.c.}}

% some specific shortcuts
\NewDocumentCommand\w{sr()}{\IfBooleanTF{#1}{\Parenthesis{#2}}{\Parenthesis*{#2}}}
\let\er\ket
\let\ar\bra
\let\era\ketbra
\let\are\braket
\NewDocumentCommand\en{so}{%
$\IfValueTF{#2}{#2}{n} \in \mathbb{N}\IfBooleanTF{#1}{_0}{}$\xspace%
}
\NewDocumentCommand\hr{mo}{\href{#1}{\IfValueTF{#2}{#2}{#1}}}

% specific quantum computing stuff
\NewDocumentCommand\cz{o}{\mathrm{CZ}\IfValueTF{#1}{_{#1}}{}}
\NewDocumentCommand\cx{o}{\mathrm{CX}\IfValueTF{#1}{_{#1}}{}}
\NewDocumentCommand\cy{o}{\mathrm{CY}\IfValueTF{#1}{_{#1}}{}}
\NewDocumentCommand\Swap{o}{\mathrm{SWAP}\IfValueTF{#1}{_{#1}}{}}
\NewDocumentCommand\iSwap{o}{\mathrm{iSWAP}\IfValueTF{#1}{_{#1}}{}}
% Pauli and Clifford group
\newcommand*\gp{\mathcal{P}}
\newcommand*\op{\overline{\mathcal{P}}}
\newcommand*\ip{\mathdutchcal{P}}
\newcommand*\uip{\hat{\mathdutchcal{P}}}
\newcommand*\oip{\overline{\mathdutchcal{P}}}
\newcommand*\ouip{\overline{\hat{\mathdutchcal{P}}}}
\newcommand*\gpr{\mathrm{P}}
\newcommand*\gc{\mathcal{C}}
\newcommand*\oc{\overline{\mathcal{C}}}
\newcommand*\ic{\mathdutchcal{C}}
\newcommand*\uic{\hat{\mathdutchcal{C}}}
\newcommand*\oic{\overline{\mathdutchcal{C}}}
\newcommand*\ouic{\overline{\hat{\mathdutchcal{C}}}}
\newcommand*\gcr{\mathrm{C}} % Clifford group representatives
\newcommand*\otau{\overline{\tau}}


\addbibresource{literature.bib}

\begin{document}

\thispagestyle{plain}
\vspace*{-1cm}
\noindent%
\begin{minipage}[t]{.7\textwidth}
  {\LARGE\scshape Conjugating Paulis with Cliffords}\\[0.2em]
\end{minipage}%
\begin{minipage}[t]{0.3\textwidth}
  \strut\hfill%
  \today%
\end{minipage}%

\tableofcontents

\addsec{Introduction}

In this document we explicitly list the conjugation rules of Clifford operators on Pauli
operators. \Cref{s"preliminaries} contains the basics for a rigorous description and
\cref{s"conjagating_paulis} the lists the conjugation rules, making use of the relation
between Clifford operators and the symplectic group, which directly represents how the
library is implemented.

\section{Preliminaries}\label{s"preliminaries}

In this section we document some well known properties of Pauli operator and Clifford
operators. It is maybe unnecessary technical, but I think this way it ensures clarity.
\begin{definition}\label{.computational_basis}
For an index set $I$, with $\abs{I} = n \in \N$, let $\w(\er{0}_i, \er{1}_i)$ be an
orthonormal basis of $\bh_{,i} \cong \C^2$ and $\sigma : \bc{0, \ldots, n-1} \to I$
bijective. The states
\begin{equation}
  \bigotimes_{i \in I} \er{x_i}_i \eqqcolon \er{x_{\sigma(n-1)} \ldots x_{\sigma(0)}}
  \eqqcolon \er{x} \quad \text{with }\; x = \sum_{i=0}^{n-1}x_{\sigma(i)}2^i
\end{equation}
for $x_i \in \bc{0, 1}$, $i \in I$ ($x \in \bc{0, \ldots, 2^{n}-1}$), define the
standard orthonormal computational basis of $\bh_n = \bigotimes_{i \in I} \bh_{,i} \cong
\C^{2^n}$. For $n = 0$, we just have the state $\er{0}$.\\ For \en*[m, n] we define
the canonical matrix representation
\begin{equation}
  \mathrm{mat}_{m,n} : \ls\w(\bh_m, \bh_n) \to \C^{2^m \times 2^n},\; A \mapsto
  \matp{
  \are{0}[A][0] & \cdots & \are{0}[A][2^m-1]\\
  \vdots & & \vdots\\
  \are{2^n-1}[A][0] & \cdots & \are{2^n-1}[A][2^m-1]
  }\,.
\end{equation}
\end{definition}
\begin{remark}\label{.label_freedom}
The freedom of $\sigma$ just means that we can relabel and resort the qubits, in a
closed context, which will be handy later on. However, usually, the index set is just
$\bc{0, \ldots n-1}$, \en*, (maybe shifted by $1$, depending on the context) and when we
define operators we will always choose the identity for $\sigma$, i.e., the qubits are
sorted from highest to lowest (left to right), if not otherwise said.
\end{remark}
\begin{definition_proposition}
Let $\M_n(R)$ be the multiplicative monoid of all $n \times n$ matrices over a ring $R$,
\en, and $\GL_n(R) \subseteq \M_n(R)$ the group of invertible matrices. We define the
conjugation action of $\GL_n(R)$ on $\M_n(R)$ via
\begin{equation}
  * : \GL_n(R) \times \M_n(R) \to \M_n(R),\; (A, B) \mapsto A*B = ABA\ui \;.
\end{equation}
For all $A \in \GL_n(R)$, the mapping
\begin{equation}
  \inn_A : M_n(R) \to \M_n(R),\; B \mapsto A * B
\end{equation}
is an automorphism of the $\M_n(R)$.
\end{definition_proposition}
\begin{notation}
It is multiplication before conjugation, i.e., $AB * CD = (AB) * (CD)$.
\end{notation}
\begin{definition}
We define $\U\w(\C^n)$, \en, to be the multiplicative group of unitary
operators on $\C^n$, and write $\U\w(n) = \U\w(\C^n)$.
\end{definition}
\begin{definition}
The Pauli operators are defined by
\begin{equation}
  X = \matp{0&1\\1&0}\;, \qquad
  Y = \matp{0&-i\\i&0}\;, \qquad
  Z = \matp{1&0\\0&-1}\;.
\end{equation}
Alternatively, we write $\sigma_x = \sigma_1 = X, \sigma_y = \sigma_2 = Y, \sigma_z =
\sigma_3 = Z$ (sometimes with upper indices), respectively, and set $\bm{\sigma} =
\w(\sigma_x\;\, \sigma_y\;\, \sigma_z)\ut$. We also consider the identity as Pauli
operator and set $\sigma_0 = \1$. Usually, when indexing $\sigma$ with Latin indices, we
count from $1$ to $3$, and when using Greek indices we count from $0$ to $3$.
\end{definition}
\begin{definition}
Let \en. The unitary infinite Pauli group $\uip_n \leq \GL\w(\bh_n)$ is defined by
\begin{equation}
  \uip_n = \set{u \, \bigotimes_{j = 1}^n \sigma_{\mu_j}}{\mu_j \in \bc{0, \ldots,
  3};\, u \in \U(1)} \;.
\end{equation}
\end{definition}
\begin{definition}
Let \en. The unitary infinite Clifford group $\uic_n \leq \U\w(\bh_n)$ is the normalizer
of the Pauli group, i.e., $\uic_n * \uip_n = \uip_n$.
\end{definition}
\begin{definition}
Let \en. We define the quotient groups
\begin{equation}
  \ouip_n = \uip/\U(1)\;, \quad \ouic_n = \uic/\U(1)\;.
\end{equation}
and set $\gpr_n$ and $\gcr_n$ to be sets of representatives, respectively.
\end{definition}
\begin{proposition}
A unitary $U \in \U\w(\bigotimes_{i = 1}^n \w(\C^2)_i)$, \en, is uniquely defined, up to
a phase, by its conjugation of the Pauli operators $X_1, Z_1, \ldots X_n, Z_n$.
\end{proposition}
\begin{proposition}\label{.tableau_description}
Let \en. $\ouip_n$ is isomorph to the abelian group $\overline{H\w(\Z_2^n)} =
\w(\Z_2^n \times \Z_2^n, +)$ with the standard addition via
\begin{align}
  \otau: \w(\Z_2^n \times \Z_2^n) &\to \ouip_n,\\ (z, x) &\mapsto \bigotimes_{j=1}^n
  \overline{Z}_j^{z_j} \overline{X}_j^{x_j}\;.
\end{align}
\end{proposition}
\begin{theorem}[\cite{bolt_clifford_stuff_1, bolt_clifford_stuff_2}]
Let \en. The Clifford group, up to Pauli operators, is isomorph to the symplectic group,
i.e.,
\begin{equation}
  \ouic_n/\ouip_n \cong \Sp_{2n}\w(\Z_2)\;,
\end{equation}
where $\Sp_{2n}\w(\Z_2)$ is the symplectic group of the $\Z_2^{2n}$ vector space with
respect to the $\smatp{0&\1\\-\1&0}$ symplectic form. More specifically, the isomorphism
is defined by
\begin{equation}
  \kappa : \ouic_n/\ouip_n \to \Sp_{2n}\w(\Z_2), \; \overline{g}\oip_n \mapsto S_g =
  \otau\ui \circ \inn_g \circ \otau\;,
\end{equation}
% where $\otau$ is defined as in \cref{.quotient_tableau_description}.
\end{theorem}
\begin{corollary}
The number of Clifford operators modulo Pauli operators is given by
\begin{equation}
  \abs{\ouic_n/\ouip_n} = 2^{n^2} \prod_{i=1}^n\w(2^{2i} - 1)
\end{equation}
\end{corollary}
\begin{theorem}
Let \en. Then it is
\begin{equation}
  \uic_n = \groupset{u, H_i, S_i, \cz[ij]}{u \in \U(1);\; i, j \in \bc{1, \ldots n}; \;
  i \neq j}\;.
\end{equation}
\end{theorem}

\section{Conjugating Paulis}\label{s"conjagating_paulis}

\begin{remark}
Note that is sufficient to calculate the conjugation rules of representatives of
$\ouic_n/\ouip_n$, \en, since Pauli operators only change the phase factor of the result
(which is trivial to calculate).
\end{remark}

\subsection{Single qubit Cliffords}

\begin{proposition}
The following mappings list all elements of $\ouic_1/\ouip_1$ (representatives) and
explicitly describe the isomorphism $\ouic_1/\ouip_1 \cong \Sp_{2}\w(\F_2)$.
\begin{table}[H]\label{t"single_qubit_isomorphism}
\center
\caption
[Explicit description of the $\ouic_1/\ouip_1 \cong \Sp_{2}\w(\F_2)$ isomophism for
every element]
{Explicit description of the $\ouic_1/\ouip_1 \cong \Sp_{2}\w(\F_2)$ isomophism for every
element: The second column specifies the element in $\ouic_1/\ouip_1$ with respect to
the generators $S$ and $H$. The first column shows the matrix description of the
canonical representatives. The third column shows the symplectic matrix according to the
isomorphism. The fourth column contains the additional phases one has when conjugating
with representatives (left for conjugating $Z$, right for conjugating $X$).}
\begin{tabular}{lccr}
  \toprule
  repr. & $\ouic_1/\ouip_1$ & $\Sp_{2}\w(\F_2)$ & phase\\
  \midrule
  $\matp{1&0\\0&1}$ & $\1$ & $\matp{1&0\\0&1}$ & $\matp{1&1}$\\
  \midrule
  $\matp{1&0\\0&i}$ & $S$ & $\matp{1&1\\0&1}$ & $\matp{1&-i}$\\
  \midrule
  $\matp{1&1\\1&-1}$ & $H$ & $\matp{0&1\\1&0}$ & $\matp{1&1}$\\
  \midrule
  $\matp{1&1\\i&-i}\frac{1}{\sqrt{2}}$ & $SH$ & $\matp{1&1\\1&0}$ & $\matp{-i&1}$\\
  \midrule
  $\matp{1&i\\1&-i}\frac{1}{\sqrt{2}}$ & $HS$ & $\matp{0&1\\1&1}$ & $\matp{1&i}$\\
  \midrule
  $\matp{1&i\\i&1}\frac{1}{\sqrt{2}}$ & $SHS$ & $\matp{1&0\\1&1}$ & $\matp{-i&1}$\\
  \bottomrule
\end{tabular}
\end{table}
\end{proposition}
\begin{proof}
We calculate the according conjugations with the representatives:
\begin{align}
  S*Z &= Z\\
  S*X &= Y = -iZX\\
  H*Z &= X\\
  H*X &= Z\\
  SH*Z &= S*X = -iZX\\
  SH*X &= S*Z = Z\\
  HS*Z &= H*Z = X\\
  HS*X &= -iH*ZX = iZX\\
  SHS*Z &= S*X = -iZX\\
  SHS*X &= iS*ZX = X
\end{align}
\end{proof}
\begin{remark}
In the following we list some typical Clifford gates and how they relate to the
representative elements of $\ouic_1/\ouip_1$ ($H^{ab}$ denotes the hermitian change from
the eigenbasis of $a$ to the eigenbasis of $b$, $a, b \in \bc{X, Y, Z}$):
\begin{align}
  S\ur &= SZ = \matp{1&0\\0&-i}\\
  HSH &= \sqrt{i}SHS = \frac{1}{\sqrt{2}}\matp{\sqrt{i}&\sqrt{-i}\\\sqrt{-i}&\sqrt{i}}\\
  \sqrt{X} &= HSH = \sqrt{i}SHS = \frac{1}{2}\matp{1+i&1-i\\1-i&1+i}\\
  \sqrt{X}\ur &= HSZH = HSHX = \sqrt{i}SHSX = \frac{1}{2}\matp{1-i&1+i\\1+i&1-i}\\
  \sqrt{Y} &= \sqrt{i}HZ = \frac{1}{2}\matp{1+i&-1-i\\1+i&1+i}\\
  \sqrt{Y}\ur &= \sqrt{-i}ZH = \sqrt{-i}HX = \frac{1}{2}\matp{1-i&1-i\\-1+i&1-i}\\
  H^{xy} &= \ee^{-i\pi/4}SX = \frac{1}{\sqrt{2}}\matp{0&1-i\\1+i&0}\\
  H^{yz} &= SHSZ = \frac{1}{\sqrt{2}}\matp{1&-i\\i&-1}\\
\end{align}
Sorted according to \cref{t"single_qubit_isomorphism} we have the following $24$
single qubit Clifford operators:
\begin{align}
  \begin{array}{llll}
    \1 = \1 & \1X = X & \1Y = Y & \1Z = Z\\
    S = S & SX = \sqrt{i}H^{xy} & SY = . & SZ = S\ur\\
    H = H & HX =\sqrt{i}\sqrt{Y}\ur & HY = . & HZ = \sqrt{-i}\sqrt{Y}\\
    SH = . & SHX = . & SHY = . & SHZ = .\\
    HS = . & HSX = . & HSY = . & HSZ = .\\
    SHS = \sqrt{-i}\sqrt{X} & SHSX = \sqrt{-i}\sqrt{X}\ur & SHSY = . & SHSZ = H^{yz}
  \end{array}
\end{align}
\end{remark}

\subsection{Two qubit Cliffords}

\begin{remark}
For the single qubit Cliffords, it made sense to define a set of canonical
representatives through the generators $S$ and $H$ (since there are only $6$ up to
Paulis). Doing the same thing for two qubit Cliffords gets out of hands; the generator
strings would be just too long. Instead we try to choose the most common ones.
\end{remark}
\begin{proposition}
The following mappings list some elements of
$\left.\w(\ouic_2/\ouip_2)\middle\backslash\w(\ouic_1/\ouip_1)\right.$ (representatives)
and explicitly describe the isomorphism $\ouic_2/\ouip_2 \cong \Sp_{4}\w(\F_2)$.
\begin{table}[H]\label{t"two_qubit_isomorphism}
\center
\caption
[Explicit description of the $\ouic_2/\ouip_2 \cong \Sp_{4}\w(\F_2)$ isomophism for some
elements]
{Explicit description of the $\ouic_2/\ouip_2 \cong \Sp_{4}\w(\F_2)$ isomophism for some
element: cf. \cref{t"single_qubit_isomorphism}. We defined in
\cref{.computational_basis,.label_freedom,.tableau_description} uniquely how we
represent the operators in matrix form: For the repr. matrix the sorted basis is
$\w(\er{0}_2\er{0}_1, \er{0}_2\er{1}_1, \er{1}_2\er{0}_1, \er{1}_2\er{1}_1)$ and for the
symplectic matrix it is $\w(Z_1, Z_2, X_1, X_2)$, under the according isomorphism,
respectively. The $\mathrm{ZC}...$ operators denote $Z$ controlled operators (i.e., a
Hadamard conjugation on the control bit with respect to the $\mathrm{C}...$ operator).}
\begin{tabular}{lccr}
  \toprule
  repr. & $\ouic_2/\ouip_2$ & $\Sp_{4}\w(\F_2)$ & phase\\
  \midrule
  $\matp{1&0&0&0\\0&1&0&0\\0&0&1&0\\0&0&0&-1}$ & $\cz$ &
  $\matp{1&0&0&1\\0&1&1&0\\0&0&1&0\\0&0&0&1}$ & $\matp{1&1&1&1}$\\
  \midrule
  $\matp{1&0&0&0\\0&1&0&0\\0&0&0&1\\0&0&1&0}$ & $\cx_{21}$ &
  $\matp{1&0&0&0\\1&1&0&0\\0&0&1&1\\0&0&0&1}$ & $\matp{1&1&1&1}$\\
  \midrule
  $\matp{1&0&0&0\\0&1&0&0\\0&0&0&-i\\0&0&i&0}$ & $\cy_{21}$ &
  $\matp{1&0&0&1\\1&1&1&0\\0&0&1&1\\0&0&0&1}$ & $\matp{1&1&1&-i}$\\
  \midrule
  $\matp{1&0&0&0\\0&0&1&0\\0&1&0&0\\0&0&0&1}$ & $\Swap$ &
  $\matp{0&1&0&0\\1&0&0&0\\0&0&0&1\\0&0&1&0}$ & $\matp{1&1&1&1}$\\
  \midrule
  $\frac{1}{2}\matp{1&1&1&-1\\1&1&-1&1\\1&-1&1&1\\-1&1&1&1}$ & $\zcx$ &
  $\matp{1&0&0&0\\0&1&0&0\\0&1&1&0\\1&0&0&1}$ & $\matp{1&1&1&1}$\\
  \midrule
  $\frac{1}{2}\matp{1&-i&1&i\\i&1&-i&1\\1&i&1&-i\\-i&1&i&1}$ & $\zcy_{21}$ &
  $\matp{1&1&0&0\\0&1&0&0\\0&1&1&0\\1&0&1&1}$ & $\matp{1&-i&1&1}$\\
  \midrule
  $\matp{1&0&0&0\\0&0&i&0\\0&i&0&0\\0&0&0&1}$ & $\iSwap$ &
  $\matp{0&1&1&1\\1&0&1&1\\0&0&0&1\\0&0&1&0}$ & $\matp{1&1&-i&-i}$\\
  \bottomrule
\end{tabular}
\end{table}
\end{proposition}
\begin{proof}
In the following we set define \en[c, t], s.t., $c > t$, i.e., $c=2$ and $t=1$. When we
use \en[a, b] as indices, the order does not matter. We
calculate the according conjugations with the representatives (we also use
\cref{.more_identities}):
\begin{align}
  \cz_{ab}*X_b &= \matp{1&0\\0&Z} \matp{X&0\\0&X} \matp{1&0\\0&Z} =
  \matp{X&0\\0&-X} = Z_aX_b\\
  %
  \cx_{ct}*Z_t &= \matp{1&0\\0&X} \matp{Z&0\\0&Z} \matp{1&0\\0&X} =
  \matp{Z&0\\0&-Z} = Z_tZ_c\\
  \cx_{ct}*Z_c &= \matp{1&0\\0&X} \matp{1&0\\0&-1} \matp{1&0\\0&X} =
  \matp{1&0\\0&-1} = Z_c\\
  \cx_{ct}*X_t &= \matp{1&0\\0&X} \matp{X&0\\0&X} \matp{1&0\\0&X} =
  \matp{X&0\\0&X} = X_t\\
  \cx_{ct}*X_c &= \matp{1&0\\0&X} \matp{0&1\\1&0} \matp{1&0\\0&X} =
  \matp{0&X\\X&0} = X_tX_c\\
  %
  \cy_{ct}*Z_t &= \ldots = Z_tZ_c\\
  \cy_{ct}*Z_c &= \ldots = Z_c\\
  \cy_{ct}*X_t &= i\cy_{ct}*Z_tY_t = \ldots = X_tZ_c\\
  \cy_{ct}*X_c &= i\cy_{ct}*Z_cY_c = \ldots = -iZ_tX_tX_c\\
  %
  \Swap_{ab}Z_a &= (\cx_{ab} * \cx_{ba}) * Z_a = \cx_{ab} \cx_{ba} * (\cx_{ab} * Z_a)\\
  &= \cx_{ab} * (\cx_{ba} * Z_a) = \cx_{ab} * Z_a Z_b = Z_b\\
  \Swap_{ab}X_b &= \ldots = X_a\\
  %
  \zcx_{ab}*Z_b &= H_a\cx_{ab} * Z_b = H_a * Z_bZ_a = Z_bX_a\\
  %
  \zcy_{ct}*Z_t &= H_c\cy_{ct} * Z_t = H_c * Z_tZ_c = Z_tX_c\\
  \zcy_{ct}*Z_c &= H_c\cy_{ct} * X_c = -i H_c * Z_tX_tX_c = -iZ_tX_tZ_c\\
  \zcy_{ct}*X_t &= H_c\cy_{ct} * X_t = H_c * X_tZ_c = X_tX_c\\
  \zcy_{ct}*X_c &= H_c\cy_{ct} * Z_c = H_c * Z_c = X_c\\
  %
  \iSwap_{ab} * Z_a &= H_b \cx_{ba} \cx_{ab} H_a S_a S_b * Z_a\\
  &= H_b \cx_{ba} \cx_{ab} * X_a = H_b \cx_{ba} * X_a X_b = H_b * X_b = Z_b\\
  \iSwap_{ab} * X_a &= H_b \cx_{ba} \cx_{ab} H_a S_a S_b * X_a = -iH_b \cx_{ba} \cx_{ab}
  H_a * Z_aX_a\\
  &= iH_b \cx_{ba} * Z_aX_aX_b = iH_b * Z_aZ_bX_b = -i Z_aZ_bX_b
\end{align}
\end{proof}
\begin{remark}
In the following we list some other typical Clifford gates which only differ by Pauli
operators (including identities):
\begin{align}
  \zcz{ct} &= H_c * \cz{ct} = \cx_{tc}\\
  \zcx{ct} &= H_c * \cx{ct}\\
  \zcx{cy} &= H_c * \cy{ct}\\
  \w(\iSwap)\ur &= \iSwap Z_1Z_2 = \matp{1&0&0&0\\0&0&-i&0\\0&-i&0&0\\0&0&0&1}\;.
\end{align}
\end{remark}

\subsection{Other operations provided by the library}

The library also provides other operations, which are not conjugations, like the moving
Paulis from one qubit to another. We define the operation $\mathrm{move\_x\_to\_x}_{sd}$
to move the $X$ Pauli on the qubit $s$ to the qubit $d$. Analog we define "x to z", "z
to x" and "z to z". They are all homomorphisms, which is why it makes sense to use just
like Clifford operations when tracking Paulis. This operations are often useful in an
MBQC related context where they can be used to move dependencies (cf.
\hr{https://docs.rs/pauli_tracker/latest/pauli_tracker/}[how the T
telportation is optimized in the ``Streamed tracking'' section here]). There are also
"remove" operations
\begin{definition}
In the following we define the actions of the ``move'' operations.
\begin{table}[H]
\center
\caption[Move operations of Pauli operators]{Move operations of Pauli operators. The
operations are linear defined on the basis $\bc{Z_s, X_s, Z_d, X_d}$ (under the $\otau$
isomorphism) for some \en[s, d].}
\begin{tabular}{lr}
  \toprule
  name & operation\\
  \midrule
  $\mathrm{move\_z\_to\_z}_{sd}$ & $\begin{array}{ccc}
    Z_s&\mapsto&Z_d\\X_s&\mapsto&X_s\\Z_d&\mapsto&Z_d\\X_d&\mapsto&X_d\end{array}$\\
  \midrule
  $\mathrm{move\_z\_to\_x}_{sd}$ & $\begin{array}{ccc}
    Z_s&\mapsto&X_d\\X_s&\mapsto&X_s\\Z_d&\mapsto&Z_d\\X_d&\mapsto&X_d\end{array}$\\
  \midrule
  $\mathrm{move\_x\_to\_z}_{sd}$ & $\begin{array}{ccc}
    Z_s&\mapsto&Z_s\\X_s&\mapsto&Z_d\\Z_d&\mapsto&Z_d\\X_d&\mapsto&X_d\end{array}$\\
  \midrule
  $\mathrm{move\_x\_to\_x}_{sd}$ & $\begin{array}{ccc}
    Z_s&\mapsto&Z_s\\X_s&\mapsto&X_d\\Z_d&\mapsto&Z_d\\X_d&\mapsto&X_d\end{array}$\\
  \bottomrule
\end{tabular}
\end{table}
\end{definition}
\begin{definition}
In the following we define the actions of the ``remove'' operations.
\begin{table}[H]
\center
\caption[Remove operations for Pauli operators]{Remove operations for Pauli operators. The
operations are linear defined on the basis $\bc{Z, X}$ (under the $\otau$ isomorphism,
where $I$ is $0$).}
\begin{tabular}{lr}
  \toprule
  name & operation\\
  \midrule
  $\mathrm{remove\_z}$ & $\begin{array}{ccc}
    Z&\mapsto&I\\X&\mapsto&X\end{array}$\\
  \midrule
  $\mathrm{remove\_x}$ & $\begin{array}{ccc}
    Z&\mapsto&Z\\X&\mapsto&I\end{array}$\\
  \bottomrule
\end{tabular}
\end{table}
\end{definition}

\appendix

\section{Other useful stuff}

\begin{proposition}\label{.more_identities}
Here are some more operator identities:
\begin{align}
  \cx_{ct} &= H_t * \cz_{ct} = H_t * \cz_{tc}\\
  \cy_{ct} &= H^{yz}_t * \cz_{ct} = H^{yz}_t * \cz_{tc}\\
  \Swap_{ab} &= \cx_{ab} * \cx_{ba}\\
  \iSwap_{ab} &= H_b \cx_{ba} \cx_{ab} H_a S_a S_b
\end{align}
\end{proposition}
\begin{proof}
\begin{align}
  H_t * \cz_{ct} &= \matp{H&0\\0&H} \matp{1&0\\0&Z} \matp{H&0\\0&H} \matp{1&0\\0&X} =
  \cx_{ct}\\
  %
  \cx_{ab} * \cx_{ba}&= \matp{1&0\\0&X} \matp{1&0&0&0\\0&0&0&1\\0&0&1&0\\0&1&0&0}
  \matp{1&0\\0&X} = \matp{1&0\\0&X} \matp{1&0&0&0\\0&0&1&0\\0&0&0&1\\0&1&0&0} =
  \Swap_{ab}\\
  %
  2H_b\cx_{ba}\cx_{ab}H_aS_aS_b &=
  \matp{1&0&0&1\\1&0&0&-1\\0&1&1&0\\0&-1&1&0} \matp{1&1\\X&-X} \matp{S&0\\0&iS}\\
  &= \matp{1&0&0&1\\1&0&0&-1\\0&1&1&0\\0&-1&1&0}
  \matp{1&0&i&0\\0&i&0&-1\\0&i&0&1\\1&0&-i&0} = \matp{2&0&0&0\\0&0&2i&0\\0&2i&0&0\\0&0&0&2} = 2 \iSwap
\end{align}
\end{proof}

\printbibliography

\end{document}
